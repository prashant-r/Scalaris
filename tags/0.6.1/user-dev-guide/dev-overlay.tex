\chapter{Basic Structured Overlay}

\section{Ring Maintenance}

\section{T-Man}

\section{Routing Tables}
\label{chapter.routing}
\svnrev{r4005}
\erlmoduleindex{rt\_beh}

Each node of the ring can perform searches in the overlay.

A search is done by a lookup in the overlay, but there are several
other demands for communication between peers. \scalaris{} provides
a general interface to route a message to the (other) peer, which is
currently responsible for a given \code{key}.

\codesnippet{api_dht_raw.erl}{api_dht_raw:lookup}{../src/api_dht_raw.erl}

The message \code{Msg} could be a \code{get_key} which retrieves content from
the responsible node or a \code{get_node} message, which returns a pointer
to the node.

All currently supported messages are listed in the file \code{dht_node.erl}.

The message routing is implemented in \code{dht_node_lookup.erl}

\codesnippet{dht_node_lookup.erl}{dht_node_lookup:routing}{../src/dht_node_lookup.erl}

Each node is responsible for a certain key interval. The function
\erlfun{intervals}{in}{/2} is used to decide, whether the key is between
the current node and its successor. If that is the case, the final step is
delivers a \code{lookup_fin} message to the local node. Otherwise, the message
is forwarded to the next nearest known peer (listed in the routing table)
determined by \erlfun{?RT}{next\_hop}{/2}.

\code{rt_beh.erl} is a generic interface for routing tables. It
can be compared to interfaces in Java. In Erlang interfaces can be
defined using a so called `behaviour'.  The files \code{rt_simple} and
\code{rt_chord} implement the behaviour `rt\_beh'.

The macro \code{?RT} is used to select the current implementation of routing
tables. It is defined in \code{include/scalaris.hrl}.

\codesnippet{scalaris.hrl}{scalaris:rt}{../include/scalaris.hrl}

The functions, that have to be implemented for a routing mechanism are
defined in the following file:

\codesnippet{rt_beh.erl}{rt_beh:behaviour}{../src/rt_beh.erl}

\begin{description}
\setlength{\parskip}{0pt}
\setlength{\itemsep}{0pt}
\erlfunindex{rt\_beh}{empty}
\item \code{empty/1} gets a successor and generates an empty routing
  table for use inside the routing table implementation. The data structure of
  the routing table is undefined. It can be a list, a tree, a matrix \ldots

\erlfunindex{rt\_beh}{empty\_ext}
\item \code{empty_ext/1} similarly creates an empty external routing table
  for use by the \erlmodule{dht\_node}. This process might not need all the
  information a routing table implementation requires and can thus work with
  less data.

\erlfunindex{rt\_beh}{hash\_key}
\item \code{hash_key/1} gets a key and maps it into the overlay's
  identifier space.

\erlfunindex{rt\_beh}{get\_random\_node\_id}
\item \code{get_random_node_id/0} returns a random node id from the
  overlay's identifier space. This is used for example when a new node
  joins the system.

\erlfunindex{rt\_beh}{next\_hop}
\item \code{next_hop/2} gets a \erlmodule{dht\_node}'s state (including the
  external routing table representation) and a key and returns the node, that
  should be contacted next when searching for the key, i.e. the known node
  nearest to the id.

\erlfunindex{rt\_beh}{init\_stabilize}
\item \code{init_stabilize/2} is called periodically to rebuild the
  routing table. The parameters are the identifier of the node, its
  successor and the old (internal) routing table state. This method may send
  messages to the \code{routing_table} process which need to be handled by
  the \code{handle_custom_message/2} handler since they are
  implementation-specific.

\erlfunindex{rt\_beh}{update}
\item \code{update/7} is called when the node's ID, predecessor and/or
  successor changes. It updates the (internal) routing table with the (new)
  information.

\erlfunindex{rt\_beh}{filter\_dead\_node}
\item \code{filter_dead_node/2} is called by the failure detector and tells
  the routing table about dead nodes. This function gets the (internal) routing
  table and a node to remove from it. A new routing table state is returned.

\erlfunindex{rt\_beh}{to\_pid\_list}
\item \code{to_pid_list/1} get the PIDs of all (internal) routing table
  entries.

\erlfunindex{rt\_beh}{get\_size}
\item \code{get_size/1} get the (internal or external) routing table's size.

\erlfunindex{rt\_beh}{get\_replica\_keys}
\item \code{get_replica_keys/1} Returns for a given (hashed)
  \code{Key} the (hashed) keys of its replicas. This used for implementing
  symmetric replication.

\erlfunindex{rt\_beh}{n}
\item \code{n/0} gets the number of available keys. An implementation may
  throw \code{throw:not_supported} if the operation is unsupported by
  the routing table.

\erlfunindex{rt\_beh}{dump}
\item \code{dump/1} dump the (internal) routing table state for debugging,
  e.g. by using the web interface. Returns a list of
  \code{\{Index, Node_as_String\}} tuples which may just as well be empty.

\erlfunindex{rt\_beh}{to\_list}
\item \code{to_list/1} convert the (external) representation of the routing
  table inside a given \code{dht_node_state} to a sorted list of known nodes
  from the routing table, i.e. first=succ, second=next known node on the ring,
  \ldots This is used by bulk-operations to create a
  broadcast tree.

\erlfunindex{rt\_beh}{export\_rt\_to\_dht\_node}
\item \code{export_rt_to_dht_node/2} convert the internal routing table state
  to an external state. Gets the internal state and the node's neighborhood
  for doing so.

\erlfunindex{rt\_beh}{handle\_custom\_message}
\item \code{handle_custom_message/2} handle messages specific to the routing
  table implementation. \erlmodule{rt\_loop} will forward unknown messages
  to this function.

\erlfunindex{rt\_beh}{check}
\item \code{check/5}, \code{check/6} check for routing table changes and send
  an updated (external) routing table to the \erlmodule{dht\_node} process.

\erlfunindex{rt\_beh}{check\_config}
\item \code{check_config/0} check that all required configuration parameters
  exist and satisfy certain restrictions.

\erlfunindex{rt\_beh}{wrap\_message}
\item \code{wrap_message/1} wraps a message send via a \code{dht_node_lookup:lookup_aux/4}.

\erlfunindex{rt\_beh}{unwrap\_message}
\item \code{unwrap_message/2} unwraps a message send via \code{dht_node_lookup:lookup_aux/4}
previously wrapped by \code{wrap_message/1}.

\end{description}

\subsection{The routing table process (\texorpdfstring{\code{rt_loop}}{rt\_loop})}
\erlmoduleindex{rt\_loop}

The \erlmodule{rt\_loop} module implements the process for all routing tables.
It processes messages and calls the appropriate methods in the specific routing
table implementations.

\codesnippet{rt_loop.erl}{rt_loop:state}{../src/rt_loop.erl}
If initialized, the node's id, its predecessor, successor and the routing table
state of the selected implementation (the macro \code{RT} refers to).

\codesnippet{rt_loop.erl}{rt_loop:trigger}{../src/rt_loop.erl}
Periodically (see \code{pointer_base_stabilization_interval} config parameter) a \code{trigger}
message is sent to the \code{rt_loop} process that starts the periodic
stabilization implemented by each routing table.

\codesnippet{rt_loop.erl}{rt_loop:update_rt}{../src/rt_loop.erl}
Every time a node's neighborhood changes, the \erlmodule{dht\_node} sends an
\code{update_rt} message to the routing table which will call
\erlfun{?RT}{update}{/7} that decides whether the routing table should be
re-build. If so, it will stop any waiting trigger and schedule an immideate
(periodic) stabilization.

\subsection{Simple routing table (\texorpdfstring{\code{rt_simple}}{rt\_simple})}
\erlmoduleindex{rt\_simple}

One implementation of a routing table is the \code{rt_simple}, which routes
via the successor. Note that this is inefficient as it needs a linear number
of hops to reach its goal. A more robust implementation, would use a successor
list. This implementation is also not very efficient in the presence of churn.

\subsubsection{Data types}
First, the data structure of the routing table is defined:

\codesnippet{rt_simple.erl}{rt_simple:types}{../src/rt_simple.erl}
The routing table only consists of a node (the successor). Keys in the overlay
are identified by integers $\geq 0$.

\subsubsection{A simple \texorpdfstring{\erlmodule{rm\_beh}}{rm\_beh} behaviour}

\erlfunindex{rt\_simple}{empty}
\codesnippet{rt_simple.erl}{rt_simple:empty}{../src/rt_simple.erl}
\erlfunindex{rt\_simple}{empty\_ext}
\codesnippet{rt_simple.erl}{rt_simple:empty_ext}{../src/rt_simple.erl}
The empty routing table (internal or external)  consists of the successor.

\erlfunindex{rt\_simple}{hash\_key}
\codesnippet{rt_simple.erl}{rt_simple:hash_key}{../src/rt_simple.erl}
Keys are hashed using MD5 and have a length of 128 bits.

\erlfunindex{rt\_simple}{get\_random\_node\_id}
\codesnippet{rt_simple.erl}{rt_simple:get_random_node_id}{../src/rt_simple.erl}
Random node id generation uses the helpers provided by the \erlmodule{randoms}
module.

\erlfunindex{rt\_simple}{next\_hop}
\codesnippet{rt_simple.erl}{rt_simple:next_hop}{../src/rt_simple.erl}
Next hop is always the successor.

\erlfunindex{rt\_simple}{init\_stabilize}
\codesnippet{rt_simple.erl}{rt_simple:init_stabilize}{../src/rt_simple.erl}
\code{init_stabilize/2} resets its routing table to the current successor.

\erlfunindex{rt\_simple}{update}
\codesnippet{rt_simple.erl}{rt_simple:update}{../src/rt_simple.erl}
\code{update/7} updates the routing table with the new successor.

\erlfunindex{rt\_simple}{filter\_dead\_node}
\codesnippet{rt_simple.erl}{rt_simple:filter_dead_node}{../src/rt_simple.erl}
\code{filter_dead_node/2} does nothing, as only the successor is listed in
the routing table and that is reset periodically in \code{init_stabilize/2}.

\erlfunindex{rt\_simple}{to\_pid\_list}
\codesnippet{rt_simple.erl}{rt_simple:to_pid_list}{../src/rt_simple.erl}
\code{to_pid_list/1} returns the pid of the successor.

\erlfunindex{rt\_simple}{get\_size}
\codesnippet{rt_simple.erl}{rt_simple:get_size}{../src/rt_simple.erl}
The size of the routing table is always \code{1}.

\erlfunindex{rt\_simple}{get\_replica\_keys}
\codesnippet{rt_simple.erl}{rt_simple:get_replica_keys}{../src/rt_simple.erl}
This \code{get_replica_keys/1} implements symmetric replication.

\erlfunindex{rt\_simple}{n}
\codesnippet{rt_simple.erl}{rt_simple:n}{../src/rt_simple.erl}
There are $2^{128}$ available keys.

\erlfunindex{rt\_simple}{dump}
\codesnippet{rt_simple.erl}{rt_simple:dump}{../src/rt_simple.erl}
\code{dump/1} lists the successor.

\erlfunindex{rt\_simple}{to\_list}
\codesnippet{rt_simple.erl}{rt_simple:to_list}{../src/rt_simple.erl}
\code{to_list/1} lists the successor from the external routing table state.

\erlfunindex{rt\_simple}{export\_rt\_to\_dht\_node}
\codesnippet{rt_simple.erl}{rt_simple:export_rt_to_dht_node}{../src/rt_simple.erl}
\code{export_rt_to_dht_node/2} states that the external routing table is the
same as the internal table.

\erlfunindex{rt\_simple}{handle\_custom\_message}
\codesnippet{rt_simple.erl}{rt_simple:handle_custom_message}{../src/rt_simple.erl}
Custom messages could be send from a routing table process on one node to the
routing table process on another node and are independent from any other
implementation.

\codesnippet{rt_simple.hrl}{rt_simple:check}{../src/rt_simple.erl}
Checks whether the routing table changed and in this case sends the
\erlmodule{dht\_node} an updated (external) routing table state. Optionally
the failure detector is updated. This may not be necessary, e.g. if
\code{check} is called after a crashed node has been reported by the failure
detector (the failure detector already unsubscribes the node in this case).
\codesnippet{rt_simple.hrl}{rt_simple:check}{../src/rt_simple.erl}

\erlfunindex{rt\_simple}{wrap\_message}
\codesnippet{rt_simple.erl}{rt_simple:wrap_message}{../src/rt_simple.erl}
Wraps a message send via \code{dht_node_lookup:lookup/4} if needed. This routing algorithm
does not need callbacks when finishing the lookup, so it does not need to wrap the
message.

\erlfunindex{rt\_simple}{unwrap\_message}
\codesnippet{rt_simple.erl}{rt_simple:unwrap_message}{../src/rt_simple.erl}
Unwraps a message previously wrapped with \code{rt\_simple:wrap_message/1}. As that
function does not wrap messages, \code{rt\_simple:unwrap_message/2} doesn't have to do
anything as well.

\subsection{Chord routing table (\texorpdfstring{\code{rt_chord}}{rt\_chord})}
\erlmoduleindex{rt\_chord}

The file \code{rt_chord.erl} implements Chord's routing.

\subsubsection{Data types}

\codesnippet{rt_chord.erl}{rt_chord:types}{../src/rt_chord.erl}

The routing table is a \code{gb_tree}. Identifiers in the ring are
integers. Note that in Erlang integer can be of arbitrary
precision. For Chord, the identifiers are in $[0, 2^{128})$,
i.e. 128-bit strings.

\subsubsection{The \texorpdfstring{\erlmodule{rm\_beh}}{rm\_beh} behaviour for Chord (excerpt)}

\erlfunindex{rt\_chord}{empty}
\codesnippet{rt_chord.erl}{rt_chord:empty}{../src/rt_chord.erl}
\erlfunindex{rt\_chord}{empty\_ext}
\codesnippet{rt_chord.erl}{rt_chord:empty_ext}{../src/rt_chord.erl}
\code{empty/1} returns an empty \code{gb_tree}, same for \code{empty_ext/1}.

\erlfun{rt\_chord}{hash\_key}{/1},
\erlfun{rt\_chord}{get\_random\_node\_id}{/0},
\erlfun{rt\_chord}{get\_replica\_keys}{/1} and
\erlfun{rt\_chord}{n}{/0} are implemented like their counterparts in
\code{rt_simple.erl}.

\erlfunindex{rt\_chord}{next\_hop}
\codesnippet{rt_chord.erl}{rt_chord:next_hop}{../src/rt_chord.erl}
If the (external) routing table contains at least one item, the next hop is
retrieved from the \code{gb_tree}. It will be the node with the largest id
that is smaller than the id we are looking for. If the routing table is empty,
the successor is chosen. However, if we haven't found the key in our routing
table, the next hop will be our largest finger, i.e. entry.

\erlfunindex{rt\_chord}{init\_stabilize}
\codesnippet{rt_chord.erl}{rt_chord:init_stabilize}{../src/rt_chord.erl}
The routing table stabilization is triggered for the first index and
then runs asynchronously, as we do not want to block the
\code{rt_loop} to perform other request while recalculating the
routing table.

We have to find the node responsible for the calculated finger and therefore
perform a lookup for the node with a \code{rt_get_node} message, including
a reference to ourselves as the reply-to address and the index to be set.

The lookup performs an overlay routing by passing the message until
the responsible node is found. There, the message is delivered to the
\erlmodule{routing\_table} process
The remote node sends the requested information back directly. It includes a
reference to itself in a \code{rt_get_node_response} message. Both messages
are handled by \erlfun{rt\_chord}{handle\_custom\_message}{/2}:

\erlfunindex{rt\_chord}{handle\_custom\_message}
\codesnippet{rt_chord.erl}{rt_chord:handle_custom_message}{../src/rt_chord.erl}
\erlfunindex{rt\_chord}{stabilize}
\codesnippet{rt_chord.erl}{rt_chord:stabilize}{../src/rt_chord.erl}

\code{stabilize/5} assigns the received routing table entry and triggers the
routing table stabilization for the the next shorter entry using the same
mechanisms as described above.

If the shortest finger is the successor, then filling the routing table is
stopped, as no further new entries would occur. It is not necessary, that
\code{Index} reaches 1 to make that happen. If less than $2^{128}$ nodes
participate in the system, it may happen earlier.

\erlfunindex{rt\_chord}{update}
\codesnippet{rt_chord.erl}{rt_chord:update}{../src/rt_chord.erl}
Tells the \erlmodule{rt\_loop} process to rebuild the routing table starting
with an empty (internal) routing table state.

\erlfunindex{rt\_chord}{filter\_dead\_node}
\codesnippet{rt_chord.erl}{rt_chord:filter_dead_node}{../src/rt_chord.erl}
\code{filter_dead_node} removes dead entries from the \code{gb_tree}.

\erlfunindex{rt\_chord}{export\_rt\_to\_dht\_node}
\codesnippet{rt_chord.erl}{rt_chord:export_rt_to_dht_node}{../src/rt_chord.erl}
\code{export_rt_to_dht_node} converts the internal \code{gb_tree} structure
based on indices into the external representation optimised for look-ups, i.e.
a \code{gb_tree} with node ids and the nodes themselves.

\codesnippet{rt_chord.hrl}{rt_chord:check}{../src/rt_chord.erl}
Checks whether the routing table changed and in this case sends the
\erlmodule{dht\_node} an updated (external) routing table state. Optionally
the failure detector is updated. This may not be necessary, e.g. if
\code{check} is called after a crashed node has been reported by the failure
detector (the failure detector already unsubscribes the node in this case).

\erlfunindex{rt\_chord}{wrap\_message}
\codesnippet{rt_chord.erl}{rt_chord:wrap_message}{../src/rt_chord.erl}

Wraps a message send via \code{dht_node_lookup:lookup/4} if needed. This
routing algorithm does not need callbacks when finishing the lookup, so it
does not need to wrap the message.

\erlfunindex{rt\_chord}{unwrap\_message}
\codesnippet{rt_chord.erl}{rt_chord:unwrap_message}{../src/rt_chord.erl}

Unwraps a message previously wrapped with
\code{rt\_chord:wrap_message/1}. As that function does not wrap messages,
\code{rt\_chord:unwrap_message/2} doesn't have to do anything as well.

\section{Local Datastore}

\section{Cyclon}

\section{Vivaldi Coordinates}

\section{Estimated Global Information (Gossiping)}

\section{Load Balancing}

\section{Broadcast Trees}
