\chapter{Setting up \scalaris{}}
\label{chapter.runscalaris}
\svnrev{r1810}

\section{Runtime Configuration}
\label{chapter.runscalaris.runtime_config}

\scalaris{} reads two configuration files from the working directory:
\code{bin/scalaris.cfg} (mandatory) and \code{bin/scalaris.local.cfg}
(optional). The former defines default settings and is included in the
release. The latter can be created by the user to alter settings.
A sample file is provided as \code{bin/scalaris.local.cfg.example}
and needs to be altered for a distributed setup (see
Section~\sieheref{user.config.distributed}).
A third way to alter the configuration of \scalaris{}, e.g. port numbers,
is to use parameters for the \code{scalarisctl} script
(ref. Section~\sieheref{user.config.scalarisctl}. The following
example changes the port to 14195 and the YAWS port to 8080:

\begin{lstlisting}[language=sh]
%> ./bin/scalarisctl -p 14194 -y 8080
\end{lstlisting}

The configuration precedence is as follows:
\begin{enumerate}
  \item configuration parameters of \code{scalarisctl}
  \item \code{bin/scalaris.local.cfg}
  \item \code{bin/scalaris.cfg}
\end{enumerate}


\subsection{Logging}
\label{sec:logging}

\scalaris{} uses the log4erl library (see \code{contrib/log4erl}) for
logging status information and error messages. The log level can be
configured in \code{bin/scalaris.cfg} for both the stdout and file logger.
The default value is {\tt warn}; only warnings, errors and severe problems are
logged.

\begin{lstlisting}[language=erlang]
%% @doc Loglevel: debug < info < warn < error < fatal < none
{log_level, warn}.
{log_level_file, warn}.
\end{lstlisting}

In some cases, it might be necessary to get more complete logging
information, e.g. for debugging. In Chapter~\sieheref{chapter.join},
we are explaining the startup process of \scalaris{} nodes in more
detail, here the {\tt info} level provides more detailed information.

\begin{lstlisting}[language=erlang]
%% @doc Loglevel: debug < info < warn < error < fatal < none
{log_level, info}.
{log_level_file, info}.
\end{lstlisting}


\section{Running \scalaris{}}


A \scalaris{} deployment can have a \emph{management server} as well as
\emph{regular nodes}. The
management server is optional and provides a global view on all nodes of a
\scalaris{} deployment which contact this server, i.e. have its address
specified in the \code{mgmt_server} configuration setting.
A regular node is either the first node in a system or joins an
existing system deployment.

\subsection{Running on a local machine}
\label{sec.boot}

Open at least two shells. In the first, inside the \scalaris{} directory,
start the first node (\code{firstnode.bat} on Windows):
\begin{lstlisting}[language=sh]
%> ./bin/firstnode.sh
\end{lstlisting}

This will start a new \scalaris{} deployment with a single node, including a
management server. On success \url{http://localhost:8000} should point to
the management interface page of the management server. The main page will
show you the number of nodes currently in the system.
A first \scalaris{} node should have started and the number should
show 1 node. The main page will also allow you to store and retrieve
key-value pairs but should not be used by applications to access
\scalaris{}. See Section~\sieheref{chapter.systemuse.apis} for application APIs.

In a second shell, you can now start a second \scalaris{} node. This
will be a `regular node':
\begin{lstlisting}[language=sh]
%> ./bin/joining_node.sh
\end{lstlisting}

The second node will read the configuration file and use this information to
contact a number of known nodes (set by the \code{known_hosts} configuration
setting) and join the ring. It will also register itself with the management
server.
The number of nodes on the web page should have increased to two by now.

Optionally, a third and fourth node can be started on the same
machine. In a third shell:
\begin{lstlisting}[language=sh]
%> ./bin/joining_node.sh 2
\end{lstlisting}

In a fourth shell:
\begin{lstlisting}[language=sh]
%> ./bin/joining_node.sh 3
\end{lstlisting}

This will add two further nodes to the deployment. The
\code{./bin/joining_node.sh} script accepts a number as its parameter which
will be added to the started node's name, i.e. \code{1} will lead to a node
named \code{node1}.
The web pages at \url{http://localhost:8000} should show the additional nodes.

\subsection{Running distributed}
\label{user.config.distributed}

\scalaris{} can be installed on other machines in the same way as
described in Section~\sieheref{sec:install}. In the default configuration,
nodes will look for the management server on \code{127.0.0.1} on port 14195.
To run \scalaris{} distributed over several nodes, each node requires a
\code{bin/scalaris.local.cfg} pointing to the node running
the management server (if available) and containing a list of known nodes.
Without a list of known nodes, a joining node will not know where to join.

In the following example, the \code{mgmt_server}'s location is defined as
an IP address plus a TCP port and its Erlang-internal process name.
If the deployment should not use a management server, replace the setting with
an invalid address, e.g. \code{'null'}.

\codesnippet{scalaris.local.cfg}{local_cfg:distributed}
            {../bin/scalaris.local.cfg.example}

If you are starting the management server using \code{firstnode.sh}, it will
listen on port 14195 and you only have to change the the IP address in the
configuration file. Otherwise the other nodes will not find the management
server. Calling \code{./bin/joining_node.sh} on a remote machine will start the
node and automatically contact the configured management server.

%\subsection{Running on PlanetLab}

%\subsection{Replication Degree}

%\subsection{Routing Scheme}

\section{Custom startup using \texorpdfstring{\code{scalarisctl}}{scalarisctl}}
\label{user.config.scalarisctl}

On Linux you can also use the \code{scalarisctl} script to start a
management server and `regular' nodes directly.
\begin{lstlisting}[language=sh]
%> ./bin/scalarisctl -h
\end{lstlisting}
\lstinputlisting[language={}]{scalarisctl-h.out}
