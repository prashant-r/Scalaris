\chapter{Download and Installation}
\label{chapter.downloadinstall}

\section{Requirements}
\label{sec.requirements}

For building and running \scalaris{}, some third-party software is
required which is not included in the \scalaris{} sources:

\begin{itemize}
\setlength{\itemsep}{0pt}
\setlength{\parskip}{0pt}
\item Erlang R13B01 or newer
\item OpenSSL (required by Erlang's crypto module)
\item GNU-like Make and autoconf (not required on Windows)
\end{itemize}

To build the Java API (and its command-line client) the following
programs are also required:

\begin{itemize}
\setlength{\itemsep}{0pt}
\setlength{\parskip}{0pt}
\item Java Development Kit 6
\item Apache Ant
\end{itemize}

Before building the Java API, make sure that \code{JAVA\_HOME} and
\code{ANT\_HOME} are set. \code{JAVA\_HOME} has to point to a JDK
installation, and \code{ANT\_HOME} has to point to an Ant installation.

To build the Python API (and its command-line client) the following
programs are also required:

\begin{itemize}
\setlength{\itemsep}{0pt}
\setlength{\parskip}{0pt}
\item Python >= 2.6
\end{itemize}

\section{Download}

The sources can be obtained from
\url{http://code.google.com/p/scalaris}. RPM and DEB packages are available
from \url{http://download.opensuse.org/repositories/home:/scalaris/} for
various Linux distributions.

\subsection{Development Branch}

You find the latest development version in the svn repository:
\begin{lstlisting}[language={}]
# Non-members may check out a read-only working copy anonymously over HTTP.
svn checkout http://scalaris.googlecode.com/svn/trunk/ scalaris-read-only
\end{lstlisting}

\subsection{Releases}

Releases can be found under the 'Download' tab on the web-page.


\section{Build}

\subsection{Linux}

\scalaris{} uses autoconf for configuring the build environment and
GNU Make for building the code.

\begin{lstlisting}[language=sh]
%> ./configure
%> make
%> make docs
\end{lstlisting}

For more details read \code{README} in the main \scalaris{} checkout
directory.

\subsection{Windows}

We are currently not supporting \scalaris{} on Windows. However, we
have two small {\tt .bat} files for building and running \scalaris{}
nodes. It seems to work but we make no guarantees.

\begin{itemize}
\item Install Erlang\\
       \url{http://www.erlang.org/download.html}
\item Install OpenSSL (for crypto module)\\
       \url{http://www.slproweb.com/products/Win32OpenSSL.html}
\item Checkout \scalaris{} code from SVN
\item adapt the path to your Erlang installation in \code{build.bat}
\item start a \code{cmd.exe}
\item go to the \scalaris{} directory
\item run \code{build.bat} in the cmd window
\item check that there were no errors during the compilation;
       warnings are fine
\item go to the bin sub-directory
\item adapt the path to your Erlang installation in \code{firstnode.bat},
       \code{joining_node.bat}
\item run \code{firstnode.bat} or one of the other start scripts in the cmd window
\end{itemize}

\code{build.bat} will generate a \code{Emakefile} if there is none yet.
If you have Erlang $<$ R13B04, you will need to adapt the \code{Emakefile}.
There will be empty lines in the first three blocks ending with
``\code{ ]\}.}'': add the following to these lines and try to compile again.
It should work now.

\begin{lstlisting}[language=erlang]
, {d, type_forward_declarations_are_not_allowed}
, {d, forward_or_recursive_types_are_not_allowed}
\end{lstlisting}

For the most recent description please see the FAQ at
\url{http://code.google.com/p/scalaris/wiki/FAQ}.

\subsection{Java-API}

The following commands will build the Java API for \scalaris{}:
\begin{lstlisting}[language=sh]
%> make java
\end{lstlisting}

This will build {\tt scalaris.jar}, which is the library for accessing
the overlay network. Optionally, the documentation can be build:
\begin{lstlisting}[language=sh]
%> cd java-api
%> ant doc
\end{lstlisting}

\subsection{Python-API}

The Python API for Python 2.* (at least 2.6) is located in the \code{python-api}
directory. Files for Python 3.* can be created using \code{2to3} from the files
in \code{python-api}. The following command will use \code{2to3} to convert the
modules and place them in \code{python3-api}. 
\begin{lstlisting}[language=sh]
%> make python3
\end{lstlisting}
Both versions of python will compile required modules on demand when executing
the scripts for the first time. However, pre-compiled modules can be created
with:
\begin{lstlisting}[language=sh]
%> make python
%> make python3
\end{lstlisting}

\subsection{Ruby-API}

The Ruby API for Ruby >= 1.8 is located in the \code{ruby-api}
directory. Compilation is not necessary.

\section{Installation}
\label{sec:install}

For simple tests, you do not need to install \scalaris{}. You can run it
directly from the source directory. Note: \code{make install} will install
\scalaris{} into \code{/usr/local} and place \code{scalarisctl} into
\code{/usr/local/bin}, by default. But it is more convenient to build an RPM
and install it.
On openSUSE, for example, do the following:

\begin{lstlisting}[language=sh]
export SCALARIS_SVN=http://scalaris.googlecode.com/svn/trunk
for package in main bindings; do
  mkdir -p ${package}
  cd ${package}
  svn export ${SCALARIS_SVN}/contrib/packages/${package}/checkout.sh
  ./checkout.sh
  cp * /usr/src/packages/SOURCES/
  rpmbuild -ba scalaris*.spec
  cd ..
done
\end{lstlisting}

If any additional packages are required in order to build an RPM,
\code{rpmbuild} will print an error.

Your source and binary RPMs will be generated in
\code{/usr/src/packages/SRPMS} and \code{RPMS}.

We build RPM and DEB packages for all tagged Scalaris versions as well as
snapshots of svn trunk and provide them using the Open Build Service.
The latest stable version is available at
\url{http://download.opensuse.org/repositories/home:/scalaris/}.
The latest svn snapshot as well as archives of previous versions are available
in their respective folders below
\url{http://download.opensuse.org/repositories/home:/scalaris:/}. Packages
are available for

\begin{itemize}
\item Fedora 16, 17,
\item Mandriva 2010, 2010.1, 2011,
\item openSUSE 11.4, 12.1, Factory, Tumbleweed
\item SLE 10, 11, 11SP1, 11SP2,
\item CentOS 5.5, 6.2,
\item RHEL 5.5, 6,
\item Debian 5.0, 6.0 and
\item Ubuntu 10.04, 10.10, 11.04, 11.10, 12.04.
\end{itemize}

An up-to-date list of available repositories can be found at
\url{https://code.google.com/p/scalaris/wiki/FAQ#Prebuild_packages}.

For those distributions which provide a recent-enough Erlang version, we build
the packages using their Erlang package and recommend using the same version
that came with the distribution. In this case we do not provide Erlang packages
in our repository.

Exceptions are made for openSUSE-based and RHEL-based distributions as well as
Debian~5.0:
\begin{itemize}
  \item For openSUSE, we provide the package from the \code{devel:languages:erlang}
repository.
  \item For RHEL-based distributions (CentOS~5, RHEL~5, RHEL~6) we included the Erlang
package from the EPEL repository of RHEL~6.
  \item For Debian~5.0 we included the Erlang package of Ubuntu 11.04.
\end{itemize}

\section{Testing the Installation}

After installing \scalaris{} you can check your installation and perform
some basic tests using

\begin{lstlisting}[language=sh]
%> scalarisctl checkinstallation
\end{lstlisting}

For further details on \code{scalarisctl} see
Section~\sieheref{user.config.scalarisctl}.
