\chapter{Introduction}

\scalaris{} is a scalable, transactional, distributed key-value store based
on the principles of structured peer-to-peer overlay networks.  It can be
used as a flexible elastic data store backend to build scalable online
services. Without system interruption it scales from a few PCs to thousands
of servers. Servers can be added or removed on the fly without any service
downtime.

\scalaris{} takes care of

\begin{center}
\begin{tabular}{lp{10cm}}
\emph{replication and fail-over} & for fault-tolerance \\
\emph{self-management} & for low maintenance overhead \\
\emph{automatic data partitioning} & for elasticity, load balancing and scalability \\
\emph{strong consistency} & to ease
development of applications on top of it, as inconsistencies have not to be
dealt with \\
\emph{transactions} & to support safe atomic updates of several data items
at once \\
\end{tabular}
\end{center}

The \scalaris{} project was initiated and is mainly developed by
\href{http://www.zib.de/DAS/}{Zuse Institute Berlin} (ZIB) and was partly
funded by the EU projects Selfman, XtreemOS, Contrail and 4CaaST. Additional
information can be found at the project homepage
(\url{http://scalaris.googlecode.com}) and the corresponding project web
page at ZIB (\url{http://www.zib.de/en/das/projekte/projektdetails/article/scalaris.html}).

The conceptual architecture of \scalaris{} consists of four layers:

\definecolor{layer_bg}{RGB}{55,96,146}
\definecolor{layer_bluefont}{RGB}{31,73,128}
\begin{center}
\begin{tikzpicture}
 [layer/.style={rectangle,fill=layer_bg,text=white,
  drop shadow={opacity=0.4},
  minimum width=7cm,
  minimum height=1.1cm,
  align=center},
  layer_desc/.style={text width=3.5cm,align=left}]
\sffamily

  %% draw the layers
  \node[layer] (tx_layer) at (0,2.7)
     {Transaction Layer};
  \node[layer,below=0.2 of tx_layer, minimum height=0.9cm] (rep_layer)
     {Replication Layer};
  \node[layer,below=0.2 of rep_layer] (son_layer)
     {Structured Overlay Network};
  \node[layer,below=0.2 of son_layer, minimum height=0.9cm] (unstructured_layer)
     {Unstructured P2P Layer};

  %% fitting the underlying bounding box
  \begin{pgfonlayer}{background}
    \node[draw,color=layer_bg,fit=(tx_layer) (rep_layer)
                                  (son_layer) (unstructured_layer),
        inner sep=0.3cm,drop shadow, fill=white] {};
  \end{pgfonlayer}

  %% text above the box
  \node[above=0.4 of tx_layer, text=layer_bg]
     {\textbf{\color{layer_bluefont}{Scalable Application using \scalaris{}}}};

  %% text below the box
  \node[inner sep=0, below=0.6 of unstructured_layer]
     {\textbf{\color{layer_bluefont}{Standard Internet Nodes for Data Storage}}};

  %% description on the right side
  \node[above right=0.3 and 0.9 of tx_layer]
     {\smaller layer implements~\ldots};
  \node[layer_desc,right=0.9 of tx_layer] (tx_layer_desc)
     {\smaller \ldots~strong consistency,
      \phantom{\ldots~}atomicity, isolation};
  \node[layer_desc,right=0.9 of rep_layer] (rep_layer_desc)
     {\smaller \ldots~availability};
  \node[layer_desc,right=0.9 of son_layer] (son_layer_desc)
     {\smaller \ldots~scalability};
  \node[layer_desc,right=0.9 of unstructured_layer] (unstructured_layer_desc)
     {\smaller \ldots~connectivity};

  %% arrows
  \begin{scope}[every node/.style={single arrow,
              fill=layer_bg,
              shape border rotate=180,
              single arrow head extend=1.2mm,
              minimum height=0.55cm,
              minimum width=0.2cm}]
    \node[right=0.15 of tx_layer] {};
    \node[right=0.15 of rep_layer] {};
    \node[right=0.15 of son_layer] {};
    \node[right=0.15 of unstructured_layer] {};
  \end{scope}
\end{tikzpicture}
\end{center}


\section{\scalaris{} provides strong consistency and partition tolerance}

In distributed computing the so called CAP theorem says that there are three
desirable properties for distributed systems, but one can only have any two
of them.

\begin{description}
\item {Strong Consistency.} Any read operation has to return the
  result of the latest write operation on the same data item.

\item {Availability.} Items can be read and modified at any time.

\item {Partition Tolerance.} The network on which the service is
  running may split into several partitions which cannot communicate
  with each other. Later on the networks may re-join again.

  For example, a service is hosted on one machine in Seattle and one
  machine in Berlin. This service is partition tolerant if it can
  tolerate that all Internet connections over the Atlantic (and
  Pacific) are interrupted for a few hours and then get repaired.
\end{description}

The goal of \scalaris{} is to provide strong consistency and partition
tolerance. We are willing to sacrifice availability to make sure that the
stored data is always consistent. I.e. when you are running \scalaris{} with
a replication degree of four and the network splits into two partitions --
one partition with three replicas and one partition with one replica -- you
will be able to continue to use the service only in the larger
partition. All requests in the smaller partition will time out or retried
until the two networks merge again. Note, most other key-value stores tend
to sacrifice consistency, which may make it hard for the application
developer to detect and handle appearing inconsistencies properly.

\section{Scientific background}

\scalaris{} is backed by tons of research. It implements both algorithms
from the literature and our own research results and combines all of them to
a practical overall system. Several aspects of \scalaris{} were analyzed
or/and developed as part of bachelor, diploma, master or PhD theses.

\subsection*{\scalaris{} in General}

{\bf Publications of the \scalaris{} team}
\begin{quote} \small
  F. Schintke. \emph{XtreemFS \& Scalaris.}
  Science \& Technology, pp. 54-55, 2013.

  A. Reinefeld, F. Schintke, T. Sch�tt, S. Haridi.
  \emph{A Scalable, Transactional Data Store for Future Internet Services.}
  Towards the Future Internet - A European Research Perspective,
  G. Tselentis et al. (Eds.) IOS Press, pp. 148-159, 2009.

  Thorsten Sch�tt, Monika Moser, Stefan Plantikow, Florian Schintke,
  Alexander Reinefeld. \emph{A Transactional Scalable Distributed Data
    Store.}  1st IEEE International Scalable Computing Challenge, co-located
  with CCGrid'08, 2008.

  Thorsten Sch�tt, Florian Schintke, Alexander Reinefeld.  \emph{Scalaris:
    Reliable Transactional P2P Key/Value Store.} ACM SIGPLAN Erlang
  Workshop, 2008.

\end{quote}

\subsection*{Structured Overlay Networks and Routing}
The general structure of \scalaris{} is modelled after Chord. The Chord
paper~\cite{chord-sigcomm} describes the ring structure, the routing
algorithms, and basic ring maintenance.

The main routines of our Chord node are in \code{src/dht\_node.erl} and the
join protocol is implemented in \code{src/dht\_node\_join.erl} (see also
Chap.~\sieheref{chapter.join}). Our implementation of the routing algorithms
is described in more detail in Sect.~\sieheref{chapter.routing} and the
actual implementation is in \code{src/rt\_chord.erl}. We also implemented
Flexible Routing Tables according to~\cite{frtchord} which can be found in
\code{src/rt\_frtchord.erl} and {src/rt\_gfrtchord.erl}.

{\bf Publications of the \scalaris{} team}
\begin{quote} \small

  Magnus M�ller. \emph{Flexible Routing Tables in a Distributed Key-Value
    Store.} Diploma thesis, HU-Berlin, 2013.

  Mikael H�gqvist. \emph{Consistent Key-Based Routing in Decentralized and
    Reconfigurable Data Services.} Doctoral thesis, HU-Berlin, 2012.

  Philipp Borgers. \emph{Erweiterung eines verteilten Key-Value-Stores
    (Riak) um einen r�umlichen Index.} Bachelor thesis, FU-Berlin, 2012.

  Thorsten Sch�tt. \emph{Range queries in distributed hash tables.} Doctoral
  thesis, 2010.

  Christian von Prollius. \emph{Ein Peer-to-Peer System mit Bereichsabfragen
    in PlanetLab.} Diploma thesis, FU-Berlin, 2008.

  Jeroen Vlek. \emph{Reducing latency: Log b routing for Chord$^{\#}$.}
  Bachelor thesis, Uni Amsterdam, 2008.

  Thorsten Sch�tt, Florian Schintke, Alexander Reinefeld.  \emph{Range
    Queries on structured overlay networks.} Computer Communications, 31(2),
  pp. 280-291, 2008.

  Thorsten Sch�tt, Florian Schintke, Alexander Reinefeld. \emph{A Structured
    Overlay for Multi-dimensional Range Queries.} Euro-Par Conference, Luc
  Anne-Marie Kermarrec (Ed.)pp. 503-513, Vol.4641, LNCS, 2007.

  Alexander Reinefeld, Florian Schintke, Thorsten Sch�tt. \emph{P2P Routing
    of Range Queries in Skewed Multidimensional Data Sets.} ZIB
  report ZR-07-23, 2007.

  Thorsten Sch�tt, Florian Schintke, Alexander Reinefeld. \emph{Structured
    Overlay without Consistent Hashing.} Sixth Workshop on Global and
  Peer-to-Peer Computing (GP2PC'06) at Sixth IEEE International Symposium on
  Cluster Computing and the Grid (CCGrid 2006), 16-19 May 2006, Singapore,
  p. 8, 2006.

  Thorsten Sch�tt, Florian Schintke, Alexander
  Reinefeld. \emph{Chord$^{\#}$: Structured Overlay Network for Non-Uniform
    Load-Distribution.} ZIB report ZR-05-40, 2005.
\end{quote}

\paragraph{Related work}
\begin{quote} \small

  \cite{frtchord} Hiroya Nagao, Kazuyuki Shudo.
  \emph{Flexible routing tables: Designing routing
    algorithms for overlays based on a total order on a routing table set.}
  In: Peer-to-Peer Computing, IEEE, 2011.

  P. Ganesan, B. Yang, H. Garcia-Molina. \emph{One torus to rule them all:
    Multi- dimensional queries in P2P systems.} In: WebDB2004, 2004.

  Luc Onana Alima, Sameh El-Ansary, Per Brand and Seif Haridi. \emph{DKS(N,
    k, f) A family of Low-Communication, Scalable and Fault-tolerant
    Infrastructures for P2P applications.} The 3rd International workshop on
    Global and P2P Computing on Large Scale Distributed Systems, (CCGRID
    2003), May 2003.

  \cite{chord-sigcomm} Ion Stoica, Robert Morris, David Karger, M. Frans
  Kaashoek and Hari Balakrishnan.  \emph{Chord: A Scalable Peer-to-peer
    Lookup Service for Internet Applications.}  ACM SIGCOMM 2001, San Deigo,
  CA, August 2001, pp. 149-160.
  \url{http://pdos.csail.mit.edu/papers/chord:sigcomm01/chord_sigcomm.pdf}
\end{quote}

\subsection*{Transactions}
The most interesting part is probably the transaction algorithms. The last
description of the algorithms and background is in~\cite{enhanced-paxos}.

The implementation consists of the Paxos algorithm in \code{src/paxos} and
the transaction algorithms itself in \code{src/transactions} (see also
Chap.~\sieheref{chapter.transactions}).

{\bf Publications of the \scalaris{} team}
\begin{quote} \small

  \cite{enhanced-paxos} Florian Schintke, Alexander Reinefeld, Seif Haridi,
  Thorsten Sch�tt.  \emph{Enhanced Paxos Commit for Transactions on DHTs.}
  CCGRID, pp. 448-454, 2010.

  Florian Schintke. \emph{Management verteilter Daten in Grid- und
    Peer-to-Peer-Systemen.} Doctoral thesis, HU-Berlin, 2010.

  Monika Moser, Seif Haridi, Tallat Shafaat, Thorsten Sch�tt, Mikael
  H�gqvist, Alexander Reinefeld. \emph{Transactional DHT Algorithms.}  ZIB
  report ZR-09-34, 2009.

  Stefan Plantikow, Alexander Reinefeld, Florian
  Schintke. \emph{Transactions and Concurrency Control for
    Peer-to-Peer-Wikis.} In: Making Grids Work, Marco Danelutto, Paraskevi
  Fragopoulo, Vladimir Getov (Eds.)pp. 337-349, 2008.

  B. Mej�as, M. H�gqvist, P. Van Roy. \emph{Visualizing Transactional Algorithms
    for DHTs.} IEEE P2P Conference, 2008.

  Monika Moser, Seif Haridi. \emph{Atomic Commitment in Transactional DHTs.}
  Proceedings of the CoreGRID Symposium, 2007.

  S. Plantikow, A. Reinefeld, F. Schintke. \emph{Distributed Wikis on
    Structured Overlays.} CoreGrid Workshop on Grid Programming Models, Grid
  and P2P System Architecture, Grid Systems, Tools and Environments, 2007.

  S. Plantikow, A. Reinefeld, F. Schintke. \emph{Transactions for
    Distributed Wikis on Structured Overlays.} DSOM, Alexander Clemm,
  Lisandro Granville, Rolf Stadler (Eds.)pp. 256-267, Vol.4785, LNCS, 2007.

  Stefan Plantikow. \emph{Transaktionen f�r verteilte Wikis auf
    strukturierten Overlay-Netzwerken.} Diploma thesis, HU-Berlin, 2007.
\end{quote}

\paragraph{Related work}
\begin{quote} \small

  Bj�rn Kolbeck, Mikael H�gqvist, Jan Stender, Felix Hupfeld. \emph{Flease
    -- Lease Coordination Without a Lock Server.} Intl. Parallel and
  Distributed Processing Symposium, pp. 978-988, 2011.

  J. Gray, L. Lamport. \emph{Consensus on transaction commit.} ACM
  Trans. Database Syst., 31(1):133--160, 2006.

  L. Lamport. \emph{Fast Paxos.} Distributed Computing, 19(2):79--103, 2006.

  L. Lamport. \emph{Paxos Made Simple.} SIGACT News, 32(4):51--58, December
  2001.

  L. Lamport. \emph{The Part-Time Parliament.} ACM Trans. Comput. Syst.,
  16(2):133--169, 1998.


\end{quote}

\subsection*{Ring Maintenance}
We changed the ring maintenance algorithm in \scalaris{}. It is not the
standard Chord one, but a variation of T-Man~\cite{t-man}. It is supposed to
fix the ring structure faster. In some situations, the standard Chord
algorithm is not able to fix the ring structure while T-Man can still fix
it. For node sampling, our implementation relies on Cyclon~\cite{cyclon}.

The T-Man implementation can be found in \code{src/rm\_tman.erl} and the
Cyclon implementation in \code{src/cyclon.erl}.

{\bf Publications of the \scalaris{} team}
\begin{quote} \small
  Paolo Costa, Guillaume Pierre, Alexander Reinefeld, Thorsten Sch�tt,
  Maarten van Steen. \emph{Sloppy Management of Structured P2P Services.}
  Proceedings of the 3$^{rd}$ International Workshop on Hot Topics in
  Autonomic Computing (HotAC III), co-located with IEEE ICAC'08, 2008.
\end{quote}

\paragraph{Related work}
\begin{quote} \small

  \cite{t-man} M{\'a}rk Jelasity, Alberto Montresor, Ozalp Babaoglu.
  \emph{T-Man: Gossip-based fast overlay topology construction.}  Computer
  Networks (CN) 53(13):2321-2339, 2009.

  \cite{cyclon} Spyros Voulgaris, Daniela Gavidia, Maarten van Steen.
  \emph{CYCLON: Inexpensive Membership Management for Unstructured P2P
    Overlays.} J. Network Syst. Manage. 13(2): 2005.

%% Tallat loopy rings, tallat thesis
\end{quote}

\subsection*{Gossiping and Topology Inference}
For some experiments, we implemented so called Vivaldi
coordinates~\cite{vivaldi}. They can be used to estimate the network latency
between arbitrary nodes.

The implementation can be found in \code{src/vivaldi.erl}.

For some algorithms, we use estimates of global
information. These estimates are aggregated with the help of gossiping
techniques~\cite{gossip}.

The implementation can be found in \code{src/gossip.erl}.

{\bf Publications of the \scalaris{} team}
\begin{quote} \small
  Marie Hoffmann. \emph{Approximate Algorithms for Distributed Systems.} Master
  thesis, FU-Berlin, 2012.

  Thorsten Sch�tt, Alexander Reinefeld, Florian Schintke, Marie Hoffmann.
  \emph{Gossip-based Topology Inference for Efficient Overlay Mapping on
    Data Centers.} Peer-to-Peer Computing, pp. 147-150, 2009.
\end{quote}

\paragraph{Related work}
\begin{quote} \small
\cite{gossip} M{\'a}rk Jelasity, Alberto Montresor, Ozalp Babaoglu.
 \emph{Gossip-based aggregation in large dynamic networks.}
 ACM Trans. Comput. Syst. 23(3), 219-252 (2005).

 \cite{vivaldi} Frank Dabek, Russ Cox, Frans Kaahoek, Robert Morris.
 \emph{Vivaldi: A Decentralized Network Coordinate System.}  ACM SIGCOMM
 2004.
\end{quote}

\subsection*{Load-Balancing}

{\bf Publications of the \scalaris{} team}
\begin{quote} \small

  Mikael H�gqvist, Nico Kruber. \emph{Passive/Active Load Balancing with
    Informed Node Placement in DHTs.} IWSOS, Thrasyvoulos Spyropoulos, Karin
  Hummel (Eds.)pp. 101-112, Vol.5918, Lecture Notes in Computer Science,
  2009.

  Nico Kruber. \emph{DHT Load Balancing with Estimated Global
    Information.} Diploma thesis, HU-Berlin, 2009.

  Mikael H�gqvist, Seif Haridi, Nico Kruber, Alexander Reinefeld, Thorsten
  Sch�tt. \emph{Using Global Information for Load Balancing in DHTs.}
  Workshop on Decentralized Self Management for Grids, P2P, and User
  Communities, 2008.

  Simon Rieche. \emph{Lastbalancierung in Peer-to-Peer Systemen.} Diploma
  thesis, FU-Berlin, 2003.
\end{quote}

\paragraph{Related work}
\begin{quote} \small

  David R. Karger, Matthias Ruhl. \emph{Simple efficient load-balancing
    algorithms for peer-to-peer systems.} Theory of Computing Systems,
  39(6):787--804, November 2006.

  Ashwin R. Bharambe, Mukesh Agrawal, Srinivasan Seshan. \emph{Mercury:
    support- ing scalable multi-attribute range queries.} SIGCOMM
  Comput. Commun. Rev., 34(4):353--366, 2004.

\end{quote}

\subsection*{Self-Management}

{\bf Publications of the \scalaris{} team}
\begin{quote} \small

T. Sch�tt, A. Reinefeld, F. Schintke, C. Hennig. \emph{Self-Adaptation in
  Large-Scale Systems.} Architectures and Languages for Self-Managing
Distributed Systems (SelfMan@SASO), 2009.

P. Van Roy, S. Haridi, A. Reinefeld, J.-B. Stefani, R. Yap,
T. Coupaye. \emph{Self Management for Large-Scale Distributed Systems.}
Formal Methods for Components and Objects 2007 (FMCO 2007), 2008.

P. Van Roy, A. Ghodsi, S. Haridi, J.-B. Stefani, T. Coupaye, A. Reinefeld,
E. Winter, R. Yap. \emph{Self Management of Large-Scale Distributed Systems
  by Combining Peer-to-Peer Networks and Components}, 2005.
\end{quote}

\subsection*{Other Topics}

{\bf Publications of the \scalaris{} team}
\begin{quote} \small

  {\bf Data Placement}

  M. H�gqvist, S. Plantikow. \emph{Towards Explicit Data Placement in
    Scalable Key/Value Stores.}  Architectures and Languages for
  Self-Managing Distributed Systems (SelfMan@SASO), 2009.

  {\bf Consistency}

  Tallat Shafaat, Monika Moser, Ali Ghodsi, Thorsten Sch�tt, Seif Haridi,
  Alexander Reinefeld. \emph{Key-Based Consistency and Availability in
    Structured Overlay Networks.}  International ICST Conference on Scalable
  Information Systems, 2008.

  Tallat Shafaat, Monika Moser, Ali Ghodsi, Thorsten Sch�tt, Alexander
  Reinefeld. \emph{On Consistency of Data in Structured Overlay Networks.}
  Coregrid Integration Workshop, 2008.

  {\bf Snapshots}

  Stefan Keidel. \emph{Snapshots in Scalaris.} Diploma thesis, HU-Berlin,
  2012.

  {\bf Replication and Replica Repair}

  Maik Lange. \emph{Redundanzverwaltung in konsistenten verteilten
    Datenbanken.} Diploma thesis, HU-Berlin, 2012.

\end{quote}

