\chapter{Testing the system}
\svnrev{r1618}

\section{Erlang unit tests}
There are some unit tests in the \code{test} directory which test \scalaris{}
itself (the Erlang code). You can call them
by running \code{make test} in the main directory. The results are stored
in a local \code{index.html} file. 

The tests are implemented with the \code{common-test} package from the
Erlang system. For running the tests we rely on \code{run\_test},
which is part of the \code{common-test} package, but (on erlang $<$ R14) is not
installed by default. \code{configure} will check whether \code{run\_test} is
available. If it is not installed, it will show a warning and a short
description of how to install the missing file.

Note: for the unit tests, we are setting up and shutting down several
overlay networks. During the shut down phase, the runtime environment
will print extensive error messages. These error messages do not
indicate that tests failed! Running the complete test suite takes
about 10-20 minutes, depending on your machine.

If the test suite is interrupted before finishing, the results may not have
been linked into the \code{index.html} file. They are however stored in the
\code{ct_run.ct@...} directory.

\section{Java unit tests}
The Java unit tests can be run by executing \code{make java-test} in the main
directory. This will start a \scalaris{} node with the default ports and test
all Java functions part of the Java API. A typical run will look like the
following:

\begin{lstlisting}[language={}]
%> make java-test
[...]
tools.test:
    [junit] Running de.zib.tools.PropertyLoaderTest
    [junit] Testsuite: de.zib.tools.PropertyLoaderTest
    [junit] Tests run: 3, Failures: 0, Errors: 0, Time elapsed: 0.113 sec
    [junit] Tests run: 3, Failures: 0, Errors: 0, Time elapsed: 0.113 sec
    [junit] 
    [junit] ------------- Standard Output ---------------
    [junit] Working Directory = <scalarisdir>/java-api/classes
    [junit] ------------- ---------------- ---------------
[...]
scalaris.test:
    [junit] Running de.zib.scalaris.ConnectionTest
    [junit] Testsuite: de.zib.scalaris.ConnectionTest
    [junit] Tests run: 7, Failures: 0, Errors: 0, Time elapsed: 0.366 sec
    [junit] Tests run: 7, Failures: 0, Errors: 0, Time elapsed: 0.366 sec
    [junit] 
    [junit] Running de.zib.scalaris.DefaultConnectionPolicyTest
    [junit] Testsuite: de.zib.scalaris.DefaultConnectionPolicyTest
    [junit] Tests run: 12, Failures: 0, Errors: 0, Time elapsed: 0.314 sec
    [junit] Tests run: 12, Failures: 0, Errors: 0, Time elapsed: 0.314 sec
    [junit] 
    [junit] Running de.zib.scalaris.PeerNodeTest
    [junit] Testsuite: de.zib.scalaris.PeerNodeTest
    [junit] Tests run: 5, Failures: 0, Errors: 0, Time elapsed: 0.077 sec
    [junit] Tests run: 5, Failures: 0, Errors: 0, Time elapsed: 0.077 sec
    [junit] 
    [junit] Running de.zib.scalaris.PubSubTest
    [junit] Testsuite: de.zib.scalaris.PubSubTest
    [junit] Tests run: 33, Failures: 0, Errors: 0, Time elapsed: 4.105 sec
    [junit] Tests run: 33, Failures: 0, Errors: 0, Time elapsed: 4.105 sec
    [junit] 
    [junit] ------------- Standard Error -----------------
    [junit] 2011-03-25 15:07:04.412:INFO::jetty-7.3.0.v20110203
    [junit] 2011-03-25 15:07:04.558:INFO::Started SelectChannelConnector@127.0.0.1:59235
    [junit] 2011-03-25 15:07:05.632:INFO::jetty-7.3.0.v20110203
    [junit] 2011-03-25 15:07:05.635:INFO::Started SelectChannelConnector@127.0.0.1:41335
    [junit] 2011-03-25 15:07:05.635:INFO::jetty-7.3.0.v20110203
    [junit] 2011-03-25 15:07:05.643:INFO::Started SelectChannelConnector@127.0.0.1:38552
    [junit] 2011-03-25 15:07:05.643:INFO::jetty-7.3.0.v20110203
    [junit] 2011-03-25 15:07:05.646:INFO::Started SelectChannelConnector@127.0.0.1:34704
    [junit] 2011-03-25 15:07:06.864:INFO::jetty-7.3.0.v20110203
    [junit] 2011-03-25 15:07:06.864:INFO::Started SelectChannelConnector@127.0.0.1:57898
    [junit] 2011-03-25 15:07:06.864:INFO::jetty-7.3.0.v20110203
    [junit] 2011-03-25 15:07:06.865:INFO::Started SelectChannelConnector@127.0.0.1:47949
    [junit] 2011-03-25 15:07:06.865:INFO::jetty-7.3.0.v20110203
    [junit] 2011-03-25 15:07:06.866:INFO::Started SelectChannelConnector@127.0.0.1:53886
    [junit] 2011-03-25 15:07:07.090:INFO::jetty-7.3.0.v20110203
    [junit] 2011-03-25 15:07:07.093:INFO::Started SelectChannelConnector@127.0.0.1:33141
    [junit] 2011-03-25 15:07:07.094:INFO::jetty-7.3.0.v20110203
    [junit] 2011-03-25 15:07:07.096:INFO::Started SelectChannelConnector@127.0.0.1:39119
    [junit] 2011-03-25 15:07:07.096:INFO::jetty-7.3.0.v20110203
    [junit] 2011-03-25 15:07:07.097:INFO::Started SelectChannelConnector@127.0.0.1:41603
    [junit] ------------- ---------------- ---------------
    [junit] Running de.zib.scalaris.ReplicatedDHTTest
    [junit] Testsuite: de.zib.scalaris.ReplicatedDHTTest
    [junit] Tests run: 6, Failures: 0, Errors: 0, Time elapsed: 0.732 sec
    [junit] Tests run: 6, Failures: 0, Errors: 0, Time elapsed: 0.732 sec
    [junit] 
    [junit] Running de.zib.scalaris.TransactionSingleOpTest
    [junit] Testsuite: de.zib.scalaris.TransactionSingleOpTest
    [junit] Tests run: 28, Failures: 0, Errors: 0, Time elapsed: 0.632 sec
    [junit] Tests run: 28, Failures: 0, Errors: 0, Time elapsed: 0.632 sec
    [junit] 
    [junit] Running de.zib.scalaris.TransactionTest
    [junit] Testsuite: de.zib.scalaris.TransactionTest
    [junit] Tests run: 18, Failures: 0, Errors: 0, Time elapsed: 0.782 sec
    [junit] Tests run: 18, Failures: 0, Errors: 0, Time elapsed: 0.782 sec
    [junit] 

test:

BUILD SUCCESSFUL
Total time: 10 seconds
'jtest_boot@csr-pc9.zib.de'
\end{lstlisting}

\section{Python unit tests}
The Python unit tests can be run by executing \code{make python-test} in the
main directory. This will start a \scalaris{} node with the default ports and test
all Python functions part of the Python API. A typical run will look like the
following:

\begin{lstlisting}[language={}]
%> make python-test
[...]
testDoubleClose (TransactionSingleOpTest.TestTransactionSingleOp) ... ok
testRead_NotConnected (TransactionSingleOpTest.TestTransactionSingleOp) ... ok
testRead_NotFound (TransactionSingleOpTest.TestTransactionSingleOp) ... ok
testTestAndSetList1 (TransactionSingleOpTest.TestTransactionSingleOp) ... ok
testTestAndSetList2 (TransactionSingleOpTest.TestTransactionSingleOp) ... ok
testTestAndSetList_NotConnected (TransactionSingleOpTest.TestTransactionSingleOp) ... ok
testTestAndSetList_NotFound (TransactionSingleOpTest.TestTransactionSingleOp) ... ok
testTestAndSetString1 (TransactionSingleOpTest.TestTransactionSingleOp) ... ok
testTestAndSetString2 (TransactionSingleOpTest.TestTransactionSingleOp) ... ok
testTestAndSetString_NotConnected (TransactionSingleOpTest.TestTransactionSingleOp) ... ok
testTestAndSetString_NotFound (TransactionSingleOpTest.TestTransactionSingleOp) ... ok
testTransactionSingleOp1 (TransactionSingleOpTest.TestTransactionSingleOp) ... ok
testTransactionSingleOp2 (TransactionSingleOpTest.TestTransactionSingleOp) ... ok
testWriteList1 (TransactionSingleOpTest.TestTransactionSingleOp) ... ok
testWriteList2 (TransactionSingleOpTest.TestTransactionSingleOp) ... ok
testWriteList_NotConnected (TransactionSingleOpTest.TestTransactionSingleOp) ... ok
testWriteString1 (TransactionSingleOpTest.TestTransactionSingleOp) ... ok
testWriteString2 (TransactionSingleOpTest.TestTransactionSingleOp) ... ok
testWriteString_NotConnected (TransactionSingleOpTest.TestTransactionSingleOp) ... ok
testAbort_Empty (TransactionTest.TestTransaction) ... ok
testAbort_NotConnected (TransactionTest.TestTransaction) ... ok
testCommit_Empty (TransactionTest.TestTransaction) ... ok
testCommit_NotConnected (TransactionTest.TestTransaction) ... ok
testDoubleClose (TransactionTest.TestTransaction) ... ok
testRead_NotConnected (TransactionTest.TestTransaction) ... ok
testRead_NotFound (TransactionTest.TestTransaction) ... ok
testTransaction1 (TransactionTest.TestTransaction) ... ok
testTransaction3 (TransactionTest.TestTransaction) ... ok
testWriteList1 (TransactionTest.TestTransaction) ... ok
testWriteString (TransactionTest.TestTransaction) ... ok
testWriteString_NotConnected (TransactionTest.TestTransaction) ... ok
testWriteString_NotFound (TransactionTest.TestTransaction) ... ok
testDelete1 (ReplicatedDHTTest.TestReplicatedDHT) ... ok
testDelete2 (ReplicatedDHTTest.TestReplicatedDHT) ... ok
testDelete_notExistingKey (ReplicatedDHTTest.TestReplicatedDHT) ... ok
testDoubleClose (ReplicatedDHTTest.TestReplicatedDHT) ... ok
testReplicatedDHT1 (ReplicatedDHTTest.TestReplicatedDHT) ... ok
testReplicatedDHT2 (ReplicatedDHTTest.TestReplicatedDHT) ... ok
testDoubleClose (PubSubTest.TestPubSub) ... ok
testGetSubscribersOtp_NotConnected (PubSubTest.TestPubSub) ... ok
testGetSubscribers_NotExistingTopic (PubSubTest.TestPubSub) ... ok
testPubSub1 (PubSubTest.TestPubSub) ... ok
testPubSub2 (PubSubTest.TestPubSub) ... ok
testPublish1 (PubSubTest.TestPubSub) ... ok
testPublish2 (PubSubTest.TestPubSub) ... ok
testPublish_NotConnected (PubSubTest.TestPubSub) ... ok
testSubscribe1 (PubSubTest.TestPubSub) ... ok
testSubscribe2 (PubSubTest.TestPubSub) ... ok
testSubscribe_NotConnected (PubSubTest.TestPubSub) ... ok
testSubscription1 (PubSubTest.TestPubSub) ... ok
testSubscription2 (PubSubTest.TestPubSub) ... ok
testSubscription3 (PubSubTest.TestPubSub) ... ok
testSubscription4 (PubSubTest.TestPubSub) ... ok
testUnsubscribe1 (PubSubTest.TestPubSub) ... ok
testUnsubscribe2 (PubSubTest.TestPubSub) ... ok
testUnsubscribe_NotConnected (PubSubTest.TestPubSub) ... ok
testUnsubscribe_NotExistingTopic (PubSubTest.TestPubSub) ... ok
testUnsubscribe_NotExistingUrl (PubSubTest.TestPubSub) ... ok

----------------------------------------------------------------------
Ran 58 tests in 12.317s

OK
'jtest_boot@csr-pc9.zib.de'
\end{lstlisting}

\section{Interoperability Tests}
In order to check whether the common types described in
Section~\sieheref{chapter.systemuse.apis} are fully supported by the APIs
and yield to the appropriate types in another API, we implemented some
interoperability tests. They can be run by executing \code{make interop-test}
in the main directory.
This will start a \scalaris{} node with the default ports, write test data using
both the Java and the Python APIs and let each API read the data it wrote
itself as well as the data the other API read. On success it will print

\begin{lstlisting}[language={}]
%> make interop-test
[...]
all tests successful
\end{lstlisting}
